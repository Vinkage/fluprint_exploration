\newglossaryentry{bu:rnaVirus}
{
    type=bus,
    name=ribonucleic acid virus(es),
    description={An \acrshort{rna} virus is a virus that has \acrshort{rna} as
    its genetic material. Inside a host cell this material is used to generate
    new virusses. Notable human diseases caused by RNA viruses include the
    common cold and influenza}
}
\newglossaryentry{bu:antigen}
{
    type=bus,
    name=antigen,
    description={In immunology, an antigen is a molecule or molecular
    structure, such as \acrshort{ha} and \acrshort{na}, that can be bound by an
    antigen-specific \gls{bu:antibody} or immune cell receptor.  The presence of
    antigens in the body normally triggers an immune response
    }
}
\newglossaryentry{bu:glycoprotein}
{
    type=bus,
    name=glycoprotein,
    description={Glycoproteins are molecules that comprise protein and
    carbohydrate chains. Many viruses have external glycoproteins that
    help them enter bodily cells, but can also serve to be important
    therapeutic or preventative targets}
}
\newglossaryentry{bu:mutation}
{
    type=bus,
    name=mutation,
    description={Mutation of genetic material occurs thanks to its chemical
    instability. The encoded protein molecules can have single amino acid
    (protein building block) change (minor, but still in many cases significant
    change leading to disease) or wide-range amino acid changes}
}
\newglossaryentry{bu:tiv}
{
    type=bus,
    name=TIV,
    description={
        An inactivated trivalent vaccine is a vaccine consisting of \gls{bu:antigen}ic virus particles from viruses that have been grown in culture and then killed to destroy disease producing capacity.
        In practice vaccines of three main types of influenza were used, hence trivalent
    },
    first={inactivated trivalent vaccines (TIV)}
}
\newglossaryentry{bu:antibody}
{
    type=bus,
    name=antibody,
    description={ Protein used by the immune system to identify and neutralize foreign objects such as pathogenic bacteria     and viruses.
    The antibody recognizes a unique molecule of the pathogen, called an \gls{bu:antigen}}
}
\newglossaryentry{bu:titer}
{
    type=bus,
    name=titer,
    description={
    Titer is a way of expressing concentration.
    Titer testing employs serial dilution to obtain approximate quantitative information from an analytical procedure that inherently only evaluates as positive or negative.
    The titer corresponds to the highest dilution factor that still yields a positive reading
    }
}
\newglossaryentry{bu:tcell}
{
    type=bus,
    name=T-cell,
    description={
        A T cell is a type of \gls{bu:lymphocyte}.
        T cells are one of the important white blood cells of the immune system and play a central role in the adaptive immune response, for example generating antibodies against influenza.
        Groups of specific, T cell subtypes have a variety of important functions in controlling and shaping the adaptive immune response
    }
}
\newglossaryentry{bu:lymphocyte}
{
    type=bus,
    name=lymphocyte,
    description={
        A lymphocyte is a type of white blood cell in the immune system of jawed vertebrates.
        Lymphocytes include \gls{bu:tcell}, and \gls{bu:bcell}.
        These cells work together in the adaptive immune response to generate antibodies against influenza
    }
}
\newglossaryentry{bu:cd8pos}
{
    type=bus,
    name=CD8+ T-cell,
    description={
        A cytotoxic T cell (also known as CD8+ T-cell) is a \gls{bu:tcell} that kills cancer cells, cells that are infected (particularly with viruses), or cells that are damaged in other ways.
        It does so by recognizing specific part of \gls{bu:antigen} and then starting a process that kills the targetted cell
    }
}
\newglossaryentry{bu:cd4pos}
{
    type=bus,
    name=CD4+ T-cell,
    description={
        The T helper cells, also known as CD4+ cells, "help" the activity of other immune cells by releasing \gls{bu:cytokine}s.
        These cells help to polarize the immune response into the appropriate kind depending on the nature of the immunological insult (e.g. virus vs. bacterium)
    }
}
\newglossaryentry{bu:cytokine}
{
    type=bus,
    name=cytokine,
    description={
        Cytokines are a broad and loose category of small proteins important in cell signaling that bind to receptor protein on the outside of (immune) cells to fulfill their signal function
    }
}
\newglossaryentry{bu:pbmc}
{
    type=bus,
    name=PBMC,
    description={
        A peripheral blood mononuclear cell is any peripheral blood cell having a round nucleus.
        These cells consist of \gls{bu:lymphocyte} and \gls{bu:monocyte}s
    },
    first={peripheral blood mononuclear cell (PBMC)}
}
\newglossaryentry{bu:bcell}
{
    type=bus,
    name=B-cell,
    description={
        B-cells produce antibody molecules; however, these antibodies are not secreted.
        Rather, they are presented on the outside of the cell where they serve as a part of B-cell receptors.
        When a B-cell is activated by an antigen, it proliferates and differentiates into an antibody-secreting effector cell, known as a plasmablast or plasma cell
    }
}
\newglossaryentry{bu:monocyte}
{
    type=bus,
    name=monocyte,
    description={
        Monocytes are a type of white blood cell.
        Monocytes and their macrophage and dendritic cell progeny serve three main functions in the immune system.
        These are phagocytosis, antigen presentation, and cytokine production.
        Phagocytosis is the process of uptake of microbes and particles followed by digestion and destruction of this material
    }
}
\newglossaryentry{bu:hai}
{
    type=bus,
    name=HAI,
    description={
        The \acrlong{ha} inhibition assay is used to measure the \gls{bu:titer} of \gls{bu:antibody} against a strain of influenza virus present in the serum.
        Antibody levels are measured before vaccination and 28 days after.
        The antibody levels are used to compute the seroprotection and seroconversion criteria
    },
    first={\acrlong{ha} inhibition assay (HAI)}
}
\newglossaryentry{bu:cmv}
{
    type=bus,
    name=CMV,
    description={
        Cytomegalovirus (CMV) is a common herpesvirus found in humans.
        Like other herpesviruses, it is a life-long infection that remains in a latent state inside the human body, until it is 'reactivated' by appropriate conditions.
        Thought to accelerate aging of the immune system and thereby impairing influenza vaccine response  \citep{van_den_Berg_2019}
    },
    first={cytomegalovirus (CMV)}
}
\newglossaryentry{bu:ebv}
{
    type=bus,
    name=EBV,
    description={
        The Epstein–Barr virus (EBV), is one of the nine known human herpesvirus types in the herpes family, and is one of the most common viruses in humans.
    },
    first={Epstein-Barr virus (EBV)}
}
\newglossaryentry{bu:seropc}
{
    type=bus,
    name=seroconversion and seroprotection,
    description={
        A vaccine is considered succesful if the recipient seroconverted (4-fold or greater rise in antibody against virus after vaccination) and were seroprotected (\acrshort{gmt} \(\ge\) 40) after vaccination.
    }
}
\newglossaryentry{bu:stat}
{
    type=bus,
    name=STAT,
    description={
        A vaccine is considered succesful if the recipient seroconverted (4-fold or greater rise in antibody against virus after vaccination) and were seroprotected (\acrshort{gmt} \(\ge\) 40) after vaccination.
    },
    first={signal transducers and activators of transcription (STAT)}
}


