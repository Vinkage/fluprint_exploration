\newglossaryentry{rnaVirus}
{
    type=bus,
    name=ribonucleic acid virus(es),
    description={An \acrshort{rna} virus is a virus that has \acrshort{rna} as
    its genetic material. Inside a host cell this material is used to generate
    new virusses. Notable human diseases caused by RNA viruses include the
    common cold and influenza}
}
\newglossaryentry{antigen}
{
    type=bus,
    name=antigen,
    description={In immunology, an antigen is a molecule or molecular
    structure, such as \acrshort{ha} and \acrshort{na}, that can be bound by an
    antigen-specific antibody or immune cell receptor.  The presence of
    antigens in the body normally triggers an immune response
    }
}
\newglossaryentry{glycoprotein}
{
    type=bus,
    name=glycoprotein,
    description={Glycoproteins are molecules that comprise protein and
    carbohydrate chains. Many viruses have external glycoproteins that
    help them enter bodily cells, but can also serve to be important
    therapeutic or preventative targets}
}
\newglossaryentry{mutation}
{
    type=bus,
    name=mutation,
    description={Mutation of genetic material occurs thanks to its chemical
    instability. The encoded protein molecules can have single amino acid
    (protein building block) change (minor, but still in many cases significant
    change leading to disease) or wide-range amino acid changes}
}
\newglossaryentry{tiv}
{
    type=bus,
    name=inactivated trivalent vaccines,
    description={An inactivated vaccine is a vaccine consisting of
    \gls{antigen}ic virus particles from viruses that have been grown in
    culture and then killed to destroy disease producing capacity. In practice
    vaccines of three main types of influenza were used, hence trivalent}
}
