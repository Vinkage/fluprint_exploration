% hello
\documentclass{article}

\usepackage[utf8]{inputenc}
\usepackage{graphicx}
\usepackage[dvipsnames]{xcolor}
\usepackage{csquotes}
\usepackage{hyperref}
\usepackage{tabularx}
\usepackage{booktabs}
\usepackage{pdfpages}
\usepackage{caption,geometry}
\usepackage[toc,page]{appendix}
\newcommand\myshade{85}
\colorlet{mylinkcolor}{violet}
\colorlet{mycitecolor}{YellowOrange}
\colorlet{myurlcolor}{Aquamarine}

\hypersetup{
  linkcolor  = mylinkcolor!\myshade!black,
  citecolor  = mycitecolor!\myshade!black,
  urlcolor   = myurlcolor!\myshade!black,
  colorlinks = true,
}

\usepackage[acronym]{glossaries}

\usepackage{listings}
\lstset{
    frame=Trbl,
    numbers=left,
    breaklines=true,
    basicstyle=\ttfamily,
    postbreak=\mbox{\textcolor{red}{$\hookrightarrow$}\space}
}

\usepackage[maxnames=3,style=authoryear,natbib=true]{biblatex}
\addbibresource{./references.bib}
% \bibliographystyle{unsrtnat}
% \setcitestyle{authoryear}


\newglossary[bsg]{bus}{bsd}{bsn}{Bussiness glossary}
\newglossary[dmg]{dm}{dmd}{dmn}{Data mining glossary}

\graphicspath{ {../images/} }

\DeclareUnicodeCharacter{2008}{-}% support older LaTeX versions
\DeclareUnicodeCharacter{2003}{ }% support older LaTeX versions

\let\oldautoref\autoref
\renewcommand{\autoref}[1]{(\oldautoref{#1})}
\newcommand{\autorefsub}[2]{(\oldautoref{#1}, #2)}
\newcommand{\gmt}{\acrshort{gmt}}
\newcommand{\firstvis}{first-visit data }
\newcommand{\secondvis}{second-visit data }

\newcommand{\uu}{Utrecht University}
\newcommand{\flup}{\gls{d:flup} }
\newcommand{\simon}{\gls{d:simon} }
\newcommand{\dpaper}{the \flup paper }
\newcommand{\Dpaper}{The \flup paper }
\newcommand{\spaper}{the \simon paper }

\newcommand{\MyTitle}[1]{
    \title{
    {#1}\\
    {\large Utrecht University}\\
    }
    \author{Mike Vink}
    \date{ \today }
    \maketitle
}
\newcommand{\f}[3]{%
\begin{figure}[htpb]
    \includegraphics[width=\textwidth]{#1}
    \caption{#2}
    \label{#3}
\end{figure}
}

\newcommand{\fptable}[5]{%

\newgeometry{scale=1}
\thispagestyle{empty}

\begin{table}
{%
    \centering
    \includegraphics[scale=.7]{#1}
    \captionsetup{width=0.8\linewidth}
    \captionof{table}{\textbf{#3} #4}
    \par
    \label{#5}
}
\end{table}

\restoregeometry
}

\newcommand{\fpfig}[5]{%

\newgeometry{scale=1}
\thispagestyle{empty}

\begin{figure}
{%
    \centering
    \includegraphics[scale=.7]{#1}
    \captionsetup{width=0.8\linewidth}
    \captionof{figure}{\textbf{#3} #4}
    \par
    \label{#5}
}
\end{figure}

\restoregeometry
}



\makeglossaries
\newglossaryentry{bu:rnaVirus}
{
    type=bus,
    name=ribonucleic acid virus(es),
    description={An \acrshort{rna} virus is a virus that has \acrshort{rna} as
    its genetic material. Inside a host cell this material is used to generate
    new virusses. Notable human diseases caused by RNA viruses include the
    common cold and influenza}
}
\newglossaryentry{bu:antigen}
{
    type=bus,
    name=antigen,
    description={In immunology, an antigen is a molecule or molecular
    structure, such as \acrshort{ha} and \acrshort{na}, that can be bound by an
    antigen-specific \gls{bu:antibody} or immune cell receptor.  The presence of
    antigens in the body normally triggers an immune response
    }
}
\newglossaryentry{bu:glycoprotein}
{
    type=bus,
    name=glycoprotein,
    description={Glycoproteins are molecules that comprise protein and
    carbohydrate chains. Many viruses have external glycoproteins that
    help them enter bodily cells, but can also serve to be important
    therapeutic or preventative targets}
}
\newglossaryentry{bu:mutation}
{
    type=bus,
    name=mutation,
    description={Mutation of genetic material occurs thanks to its chemical
    instability. The encoded protein molecules can have single amino acid
    (protein building block) change (minor, but still in many cases significant
    change leading to disease) or wide-range amino acid changes}
}
\newglossaryentry{bu:tiv}
{
    type=bus,
    name=TIV,
    description={
        An inactivated trivalent vaccine is a vaccine consisting of \gls{bu:antigen}ic virus particles from viruses that have been grown in culture and then killed to destroy disease producing capacity.
        In practice vaccines of three main types of influenza were used, hence trivalent
    },
    first={inactivated trivalent vaccines (TIV)}
}
\newglossaryentry{bu:antibody}
{
    type=bus,
    name=antibody,
    description={ Protein used by the immune system to identify and neutralize foreign objects such as pathogenic bacteria     and viruses.
    The antibody recognizes a unique molecule of the pathogen, called an \gls{bu:antigen}}
}
\newglossaryentry{bu:titer}
{
    type=bus,
    name=titer,
    description={
    Titer is a way of expressing concentration.
    Titer testing employs serial dilution to obtain approximate quantitative information from an analytical procedure that inherently only evaluates as positive or negative.
    The titer corresponds to the highest dilution factor that still yields a positive reading
    }
}
\newglossaryentry{bu:tcell}
{
    type=bus,
    name=T-cell,
    description={
        A T cell is a type of \gls{bu:lymphocyte}.
        T cells are one of the important white blood cells of the immune system and play a central role in the adaptive immune response, for example generating antibodies against influenza.
        Groups of specific, T cell subtypes have a variety of important functions in controlling and shaping the adaptive immune response
    }
}
\newglossaryentry{bu:lymphocyte}
{
    type=bus,
    name=lymphocyte,
    description={
        A lymphocyte is a type of white blood cell in the immune system of jawed vertebrates.
        Lymphocytes include \gls{bu:tcell}, and \gls{bu:bcell}.
        These cells work together in the adaptive immune response to generate antibodies against influenza
    }
}
\newglossaryentry{bu:cd8pos}
{
    type=bus,
    name=CD8+ T-cell,
    description={
        A cytotoxic T cell (also known as CD8+ T-cell) is a \gls{bu:tcell} that kills cancer cells, cells that are infected (particularly with viruses), or cells that are damaged in other ways.
        It does so by recognizing specific part of \gls{bu:antigen} and then starting a process that kills the targetted cell
    }
}
\newglossaryentry{bu:cd4pos}
{
    type=bus,
    name=CD4+ T-cell,
    description={
        The T helper cells, also known as CD4+ cells, "help" the activity of other immune cells by releasing \gls{bu:cytokine}s.
        These cells help to polarize the immune response into the appropriate kind depending on the nature of the immunological insult (e.g. virus vs. bacterium)
    }
}
\newglossaryentry{bu:cytokine}
{
    type=bus,
    name=cytokine,
    description={
        Cytokines are a broad and loose category of small proteins important in cell signaling that bind to receptor protein on the outside of (immune) cells to fulfill their signal function
    }
}
\newglossaryentry{bu:pbmc}
{
    type=bus,
    name=PBMC,
    description={
        A peripheral blood mononuclear cell is any peripheral blood cell having a round nucleus.
        These cells consist of \gls{bu:lymphocyte} and \gls{bu:monocyte}s
    },
    first={peripheral blood mononuclear cell (PBMC)}
}
\newglossaryentry{bu:bcell}
{
    type=bus,
    name=B-cell,
    description={
        B-cells produce antibody molecules; however, these antibodies are not secreted.
        Rather, they are presented on the outside of the cell where they serve as a part of B-cell receptors.
        When a B-cell is activated by an antigen, it proliferates and differentiates into an antibody-secreting effector cell, known as a plasmablast or plasma cell
    }
}
\newglossaryentry{bu:monocyte}
{
    type=bus,
    name=monocyte,
    description={
        Monocytes are a type of white blood cell.
        Monocytes and their macrophage and dendritic cell progeny serve three main functions in the immune system.
        These are phagocytosis, antigen presentation, and cytokine production.
        Phagocytosis is the process of uptake of microbes and particles followed by digestion and destruction of this material
    }
}
\newglossaryentry{bu:hai}
{
    type=bus,
    name=HAI,
    description={
        The \acrlong{ha} inhibition assay is used to measure the \gls{bu:titer} of \gls{bu:antibody} against a strain of influenza virus present in the serum.
        Antibody levels are measured before vaccination and 28 days after.
        The antibody levels are used to compute the seroprotection and seroconversion criteria
    },
    first={\acrlong{ha} inhibition assay (HAI)}
}
\newglossaryentry{bu:cmv}
{
    type=bus,
    name=CMV,
    description={
        Cytomegalovirus (CMV) is a common herpesvirus found in humans.
        Like other herpesviruses, it is a life-long infection that remains in a latent state inside the human body, until it is 'reactivated' by appropriate conditions.
        Thought to accelerate aging of the immune system and thereby impairing influenza vaccine response  \citep{van_den_Berg_2019}
    },
    first={cytomegalovirus (CMV)}
}
\newglossaryentry{bu:ebv}
{
    type=bus,
    name=EBV,
    description={
        The Epstein–Barr virus (EBV), is one of the nine known human herpesvirus types in the herpes family, and is one of the most common viruses in humans.
    },
    first={Epstein-Barr virus (EBV)}
}
\newglossaryentry{bu:seropc}
{
    type=bus,
    name=seroconversion and seroprotection,
    description={
        A vaccine is considered succesful if the recipient seroconverted (4-fold or greater rise in antibody against virus after vaccination) and were seroprotected (\acrshort{gmt} \(\ge\) 40) after vaccination.
    }
}
\newglossaryentry{bu:stat}
{
    type=bus,
    name=STAT,
    description={
        A vaccine is considered succesful if the recipient seroconverted (4-fold or greater rise in antibody against virus after vaccination) and were seroprotected (\acrshort{gmt} \(\ge\) 40) after vaccination.
    },
    first={signal transducers and activators of transcription (STAT)}
}



\newglossaryentry{d:model}
{
    type=dm,
    name=model,
    description={model is a model}
}
\newglossaryentry{d:flup}
{
    type=dm,
    name=FluPrint,
    description={Data used in this work}
}
\newglossaryentry{d:simon}
{
    type=dm,
    name=SIMON,
    description={Follow up study used in this work}
}

\newacronym{ha}{HA}{hemagglutinin}
\newacronym{na}{NA}{neuraminidase}
\newacronym{rna}{RNA}{ribonucleic acid}


\begin{document}
\MyTitle{Bussiness Understanding Report}
\tableofcontents
\printglossary[type=bus]
\printglossary[type=dm]
\printglossary[type=\acronymtype]


\section{background}

Influenza viruses are enveloped \gls{rnaVirus} (\acrshort{rna} virus(es)) and
are divided into three types on the basis of \gls{antigen}ic differences of internal
structural proteins \citep{fdaGuidanceIndustryClinical2007}.

Two influenza virus types, Type A and B, cause yearly epidemic outbreaks in humans
and are further classified based on the structure of two major external
\gls{glycoprotein}s, hemagglutinin (\acrshort{ha}) and neuraminidase (\acrshort{na})
\citep{fdaGuidanceIndustryClinical2007}.

Type B viruses, which are largely restricted to the human host, have a single
\acrshort{ha} and \acrshort{na} subtype.  In contrast, numerous \acrshort{ha}
and \acrshort{na} Type A influenza subtypes have been identified to date.  Type
A and B influenza variant strains emerge as a result of frequent
\gls{antigen}ic change, principally from \gls{mutation}s in the \acrshort{ha}
and \acrshort{na} \gls{glycoprotein}s \citep{fdaGuidanceIndustryClinical2007}.

Since 1977, influenza A virus subtypes H1N1 and H3N2, and influenza B viruses
have been in global circulation in humans. The current U.S. licensed
\gls{tiv} are formulated to prevent influenza illness
caused by these influenza viruses.  Because of the frequent emergence of new
influenza variant strains, the \gls{antigen}ic composition of influenza vaccines
needs to be evaluated yearly, and the \gls{tiv} are reformulated almost every
year.

Currently, even with full production, manufacturing capacity would not produce
enough seasonal influenza vaccine to vaccinate all those for whom the vaccine
is now recommended \citep{fdaGuidanceIndustryClinical2007}.

\subsection{Influenza mortality estimation models}

Numerous works apply regression models to describe seasonal population
influenza mortality \citep{zhouHospitalizationsAssociatedInfluenza2012,
greenMortalityAttributableInfluenza2013, iulianoEstimatesGlobalSeasonal2018}.
Reported are varying age-specific influenza burdens during different seasonal
epidemics for different regions, but in general young children an elderly are
found to be more susceptible to influenza and are adviced to vaccinated
annually \citep{zhouHospitalizationsAssociatedInfluenza2012}.

Specifically, within the US based work of
\cite{zhouHospitalizationsAssociatedInfluenza2012}, the highest hospitalization
rates for influenza were among persons aged $>=$65 years and those aged $<$1 year.
And, age-standardized annual rates per 100000 person-years varied substantially
for influenza. A similar pattern is in
\cite{greenMortalityAttributableInfluenza2013}, where an age shift in Wales and
England seasonal influenza burden was observed following the 2009 swine flue
pandemic. These patterns can confound decision making on national and
international public health policies. The necessity of informed decision making
is apperant from estimates of influenza attributed mortality, it is
estimated that globally 291.243–645.832 influenza associated seasonal deaths
occur annually \citep{iulianoEstimatesGlobalSeasonal2018}.

\subsection{Vaccine success criteria}

Due to the volume and vulnerability of population groups most at risk for
influenze, the young and the elderly, a placebo controlled vaccine efficacy
study is extremely costly \citep{zhouHospitalizationsAssociatedInfluenza2012}.
Instead the haemagglutination-inhibiting (HAI) antibody test for influenza
virus antibody is used to assess vaccine protection
\citep{dejongHaemagglutinationinhibitingAntibodyInfluenza2003}. The policy for
a succesful vaccine is an 4-fold increase in HAI antibody titre after
vaccination and a geometric mean HAI titer of $\geq$ 40. The last is predicted
to reduce influenza risk by 50\%
\cite{dejongHaemagglutinationinhibitingAntibodyInfluenza2003}.

\subsection{Finding immunological factors predisposing vaccine HAI antibody response using machine learning}

It is known that pre-existing T cell populations are correlated with a HAI
antibody response after vaccination. But, the role of T cells in mediating that
response is uncertain. In one work it was found that under certain
circumstances CD8+ T cells specific to conserved viral epitopes correlated with
protection against symptomatic influenza
\citep{sridharCellularImmuneCorrelates2013}.In other work, populations of CD4+
T cells that associated with protective antibody responses after seasonal
influenza vaccinations were found \citep{bentebibelInductionICOSCXCR3}.
\cite{trieuLongtermMaintenanceInfluenzaSpecific2017} reports a stable CD8+ T
cell response and an increased CD4+ T cell response after vaccination. It was
also reported that repeat vaccinations are an important factor in maintaining
CD4+ T cell population \citep{trieuLongtermMaintenanceInfluenzaSpecific2017}.
How exactly these T cell populations factor into protective influenza immunity
and vaccination reponse is not well understood.

Machine learning has been applied to clinical datasets to find influenza
protection markers, such as the described T cell populations and titers of
related molecules \citep{furmanApoptosisOtherImmune2013,
sobolevAdjuvantedInfluenzaH1N1Vaccination2016, tsangGlobalAnalysesHuman2014}.
These type of studies suffer from data quality issues, such as: inconsistencies
between findings depending on the epidemic season, only focussing on one type
of biological assay to get data, and a low amount of patients/samples. A
succesful vaccination is also often not well defined.

\subsection{Bussiness objectives}

Due to the high volume population that needs vaccines, it is important to study
immune correlates to vaccine response. For example, repeat vaccination might
not be necessary if the response is low, or a different vaccine is desired on a
person to person basis depending on immune correlates. Moreover, identifying
patterns between vaccine response and immune correlates furthers the
understanding of the underlying immunological mechanism of influenza
protection.

This work uses the FluPrint database, which aims to solve some of the data
quality issues of prior studies using clinical datasets comprised of blood and
serum sample assays. It does so by incorporating eigth clinical studies
conducted between 2007 to 2015 using in total 740 patients, including different
types of assays and normalizing their values, and by providing a binary
classification of high- and low-responder to a vaccine.

The objectives of this work are to answer:
\begin{itemize}
        \item Which datasets in the FluPrint database are most interesting?
        \item How do different clinical studies compare?
        \item What are the differences in efficacy between vaccination types?
        \item What is the effect of repeat vaccination on vaccine response?
        \item What immunological factors correlate to a high vaccine response?
\end{itemize}

Since this work is an independent study performed for an assignment, the
success criteria for these objective will be loosely defined as providing a
statistical description or to provide insigth in the questions posed in the
objectives.

The rationale for these questions and succes criteria  are based on the scope
of the 3EC project as part of the Applied data science profile and the data
available. The paper of \cite{tomicFluPRINTDatasetMultidimensional2019} on
which this work is mostly based on provides these questions as interesting
directions for further analysis, but does not directly provide the data
necessary to answer them, only the MySQL database containing a great volume of
data.

\section{Assess situation}

\subsection{data sources}

The only source of data used in the project is provided by
\cite{tomicFluPRINTDatasetMultidimensional2019}. It is a MySQL database for
which the installation is described in the
\href{https://github.com/LogIN-/fluprint}{FluPrint Github Repository}. A
template query is also provided by the authors on the github page belonging to
an unpublished work by the same authors
\href{https://github.com/LogIN-/simon-manuscript}{SIMON Github Repository}.
According to the authors, this data is the most interesting for the bussiness
objective of finding repeat vaccination effects and will be used in this work
too \cite{tomicSIMONAutomatedMachine2019}. The authors give this brief
description of the data:

\begin{displayquote}
The influenza datasets were obtained from the Stanford Data Miner maintained by
    the Human Immune Monitoring Center at Stanford University. This included
    total of 177 csv files, which were automatically imported to the MySQL
    database to facilitate further analysis. The database, named FluPRINT and
    its source code, including the installation tutorial are freely available
    here and on project's website. Following database installation, you can
    obtain data used in the SIMON publication by following MySQL database
    query:
\end{displayquote}

\begin{lstlisting}[language=sql, caption=Query of initial SIMON data, label={lst:QueryTemplate}]
SELECT donors.id                        AS donor_id,
       donor_visits.age                 AS age,
       donor_visits.vaccine_resp        AS outcome,
       experimental_data.name_formatted AS data_name,
       experimental_data.data           AS data
FROM   donors
       LEFT JOIN donor_visits
              ON donors.id = donor_visits.donor_id
                 AND donor_visits.visit_id = 1
       INNER JOIN experimental_data
               ON donor_visits.id = experimental_data.donor_visits_id
                  AND experimental_data.donor_id = donor_visits.donor_id
WHERE  donors.gender IS NOT NULL
       AND donor_visits.vaccine_resp IS NOT NULL
       AND donor_visits.vaccine = 4
ORDER  BY donors.study_donor_id DESC
\end{lstlisting}

\subsection{Tools and techniques}

Installation of the FluPrint database will require an installation on a
unix operating system of \href{https://www.mysql.com/}{MySQL},
\href{https://www.php.net/manual/en/install.php}{PHP}. More details are at the
\href{https://github.com/LogIN-/fluprint}{FluPrint Github Repository}.

Database querying was done using the \href{https://neovim.io/}{neovim} toolset,
personal configuration can be found
\href{https://github.com/Vinkage/mike_neovim/tree/feature}{here}.

Since the work this paper is based on uses the R toolset, it is also used here
\citep{tomicFluPRINTDatasetMultidimensional2019,
tomicSIMONAutomatedMachine2019}.  Especially crucial is the
\href{https://cran.r-project.org/web/packages/mulset/index.html}{R package
mulset}, which was made by the authors. This package is used to deal with
missing data between different clinical studies and years, and thus will be
used to generate complete data tables in this paper too. All scripts in this
work were composed using tidyverse packages in combination with modelling
packages.

\subsection{Requirements of the project}

Requirements of this work are to show ability in using data science methods.
As such, most of the insights will inevitably be a replication of the work done
by the authors of the FluPrint database \cite{tomicSIMONAutomatedMachine2019},
but all the scripts and analysis done are original work and are supplied
together with the final deliverable.

Since the data type used here is a database this makes it more complicated for
an examinator to reproduce all code, especially since installing the database
requires a unix operating system. This is not considered problematic
since the queried tables from the database will be included in the final
deliverable.

Reporting of the project follows the CRISP-DM methodology, where at each
stage of the project a separate report is written during the analysis work. In
the end the most important information is kept and incorporated in a final
report that is assumed to be graded in conjunction with the code.

\subsection{Assumptions of the project}

This work assumes that the focus point of the evaluation lies on the
methodology used, and the ability to apply the basic data science methods
learned in the Applied Data Science profile. The answer to business objectives
is assumed to be subjective, and it is assumed that the methods used and
clarity of insights into the data gained are more important.

It is also assumed that the FluPrint database and other methods used by the
authors \cite{tomicFluPRINTDatasetMultidimensional2019,
tomicSIMONAutomatedMachine2019} are of high quality, and that this is
appropriate for this work. Out of the scope of this work is investigating
whether the preprocessing done for the data in the database is valid, since we
are not domain experts. A method for querying, cleaning, and generating
complete data tables has been provided by the authors and will also be used in
this work. It is assumed that the SQL and R methods (in particular the mulset R
package) in question are allowed to be used as a starting point in this
assignment.

\subsection{Constraints of the project}

This work is an unsupervised assignment, and only personal hardware were
available. This put constraints on dataset size and computational requirements
of analyses. The work was done on a Macbook air (2017) with the OSX big-sur
operating system. This means that unix tools were available and there were no
technical constraints. The filetypes are only csv files generated by the SQL
server.

\section{Data mining goals}

All bussiness objectives described involve querying data from the FluPrint
database. The goal of the authors of the FluPrint database was to provide a
unqiue opportunity to study immune correlates of high vaccine responders across
different years and clinical studies. The authors also provide a binary
classification for donors. In this work we first and foremost explore the
database, and lastly we apply feature selection methods and classification
models on the most interesting dataset.

The bussiness objectives can be translated in data mining terminology like so:
\begin{itemize}
        \item Explore and describe SQL queries and corresponding csv tables.
        \item Model and visualise the different clinical study populations.
        \item Model and visualise the difference between vaccination types.
        \item Model and visualise repeat vaccination effects.
        \item Apply standard feature selection methods to the most interesting dataset.
        \item Fit classification models to the most interesting dataset.
\end{itemize}

In data mining terms, the problem type is a combination of exploratory data
analysis and classification. Since this work is for a 3EC assignment for the
Applied Data Science profile and most of the goals are exploratory analyses,
success criteria for all goals are subjective. For exploratory and visual type
goals the quality is expected to be of the same level as the publications of
the authors \cite{tomicFluPRINTDatasetMultidimensional2019,
tomicSIMONAutomatedMachine2019}.  For the classification type goals we follow
the model evaluation procedure used by the authors
\cite{tomicSIMONAutomatedMachine2019}, models were evaluated by the AUROC
metric, and accuracy, specificity and sensitivity were also reported. Insights
produced by this work were benchmarked against the work of the original
authors.

\section{Project plan}

\f{sql_querying_plan}
{Project plan for the SQL related data mining goal.}
{plan:sql}

The first part of the project involved querying the database, and collecting
and describing the available data \autoref{plan:sql}. The first goal is to
understand the tables in the SQL database, their key relations, and to describe
the attributes within the tables. Valuable info on this part is already
provided in the original publication of the database
\cite{tomicFluPRINTDatasetMultidimensional2019}, but it was also investigated
in this work. The tools that will be used are SQL for querying and R for
statistical descriptions.

The second phase of this plan was an iterative process of finding suitable data
to answer the modelling and visualisation data mining goals. This is a more
involved process since it requires exploration of the database to answer the
questions, and therefore was estimated to take time.

\f{model_and_vis_plan}
{Project plan for the modelling and visualisation data mining goals.}
{plan:vis}

Relations between attributes in the generated datasets are visualised and
modelled to see if there exist a pattern in the data that is relevant for the
business objectives \autoref{plan:vis}. A critical point in this plan is
deciding whether an objective cannot be answered with the available data. In
that case the goal was revised and the second phase of the SQL query plan was
reiterated. When deciding if the exploratory analysis was of sufficient
quality, the work by the authors of the database used in this work was used as
a subjective benchmark \cite{tomicSIMONAutomatedMachine2019,
tomicFluPRINTDatasetMultidimensional2019}.

\f{feature_selection_classification}
{Project plan for the classification and feature selection data mining goal.}
{plan:cls}

For the final two data mining goals the plan was to find the immune correlates
of high immune responders using a wrapper based feature selection strategy
\autoref{plan:cls}

\printbibliography
\end{document}
