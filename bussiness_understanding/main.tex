% hello
\documentclass{article}

\usepackage[utf8]{inputenc}
\usepackage{graphicx}
\usepackage[dvipsnames]{xcolor}
\usepackage{csquotes}
\usepackage{hyperref}
\usepackage{tabularx}
\usepackage{booktabs}
\usepackage{pdfpages}
\usepackage{caption,geometry}
\usepackage[toc,page]{appendix}
\newcommand\myshade{85}
\colorlet{mylinkcolor}{violet}
\colorlet{mycitecolor}{YellowOrange}
\colorlet{myurlcolor}{Aquamarine}

\hypersetup{
  linkcolor  = mylinkcolor!\myshade!black,
  citecolor  = mycitecolor!\myshade!black,
  urlcolor   = myurlcolor!\myshade!black,
  colorlinks = true,
}

\usepackage[acronym]{glossaries}

\usepackage{listings}
\lstset{
    frame=Trbl,
    numbers=left,
    breaklines=true,
    basicstyle=\ttfamily,
    postbreak=\mbox{\textcolor{red}{$\hookrightarrow$}\space}
}

\usepackage[maxnames=3,style=authoryear,natbib=true]{biblatex}
\addbibresource{./references.bib}
% \bibliographystyle{unsrtnat}
% \setcitestyle{authoryear}


\newglossary[bsg]{bus}{bsd}{bsn}{Bussiness glossary}
\newglossary[dmg]{dm}{dmd}{dmn}{Data mining glossary}

\graphicspath{ {../images/} }

\DeclareUnicodeCharacter{2008}{-}% support older LaTeX versions
\DeclareUnicodeCharacter{2003}{ }% support older LaTeX versions

\let\oldautoref\autoref
\renewcommand{\autoref}[1]{(\oldautoref{#1})}

\newcommand{\uu}{Utrecht University}
\newcommand{\MyTitle}[1]{
    \title{
    {#1}\\
    {\large Utrecht University}\\
    }
    \author{Mike Vink}
    \date{ \today }
    \maketitle
}
\newcommand{\f}[3]{%
\begin{figure}
    \includegraphics[width=\textwidth]{#1}
    \caption{#2}
    \label{#3}
\end{figure}
}

\newcommand{\fptable}[5]{%

\newgeometry{scale=1}
\thispagestyle{empty}

\begin{table}
{%
    \centering
    \includegraphics[scale=.7]{#1}
    \captionsetup{width=0.8\linewidth}
    \captionof{table}{\textbf{#3} #4}
    \par
    \label{#5}
}
\end{table}

\restoregeometry
}

\newcommand{\fpfig}[5]{%

\newgeometry{scale=1}
\thispagestyle{empty}

\begin{figure}
{%
    \centering
    \includegraphics[scale=.7]{#1}
    \captionsetup{width=0.8\linewidth}
    \captionof{table}{\textbf{#3} #4}
    \par
    \label{#5}
}
\end{figure}

\restoregeometry
}





\makeglossaries
\newglossaryentry{bu:rnaVirus}
{
    type=bus,
    name=ribonucleic acid virus(es),
    description={An \acrshort{rna} virus is a virus that has \acrshort{rna} as
    its genetic material. Inside a host cell this material is used to generate
    new virusses. Notable human diseases caused by RNA viruses include the
    common cold and influenza}
}
\newglossaryentry{bu:antigen}
{
    type=bus,
    name=antigen,
    description={In immunology, an antigen is a molecule or molecular
    structure, such as \acrshort{ha} and \acrshort{na}, that can be bound by an
    antigen-specific \gls{bu:antibody} or immune cell receptor.  The presence of
    antigens in the body normally triggers an immune response
    }
}
\newglossaryentry{bu:glycoprotein}
{
    type=bus,
    name=glycoprotein,
    description={Glycoproteins are molecules that comprise protein and
    carbohydrate chains. Many viruses have external glycoproteins that
    help them enter bodily cells, but can also serve to be important
    therapeutic or preventative targets}
}
\newglossaryentry{bu:mutation}
{
    type=bus,
    name=mutation,
    description={Mutation of genetic material occurs thanks to its chemical
    instability. The encoded protein molecules can have single amino acid
    (protein building block) change (minor, but still in many cases significant
    change leading to disease) or wide-range amino acid changes}
}
\newglossaryentry{bu:tiv}
{
    type=bus,
    name=TIV,
    description={
        An inactivated trivalent vaccine is a vaccine consisting of \gls{bu:antigen}ic virus particles from viruses that have been grown in culture and then killed to destroy disease producing capacity.
        In practice vaccines of three main types of influenza were used, hence trivalent
    },
    first={inactivated trivalent vaccines (TIV)}
}
\newglossaryentry{bu:antibody}
{
    type=bus,
    name=antibody,
    description={ Protein used by the immune system to identify and neutralize foreign objects such as pathogenic bacteria     and viruses.
    The antibody recognizes a unique molecule of the pathogen, called an \gls{bu:antigen}}
}
\newglossaryentry{bu:titer}
{
    type=bus,
    name=titer,
    description={
    Titer is a way of expressing concentration.
    Titer testing employs serial dilution to obtain approximate quantitative information from an analytical procedure that inherently only evaluates as positive or negative.
    The titer corresponds to the highest dilution factor that still yields a positive reading
    }
}
\newglossaryentry{bu:tcell}
{
    type=bus,
    name=T-cell,
    description={
        A T cell is a type of \gls{bu:lymphocyte}.
        T cells are one of the important white blood cells of the immune system and play a central role in the adaptive immune response, for example generating antibodies against influenza.
        Groups of specific, T cell subtypes have a variety of important functions in controlling and shaping the adaptive immune response
    }
}
\newglossaryentry{bu:lymphocyte}
{
    type=bus,
    name=lymphocyte,
    description={
        A lymphocyte is a type of white blood cell in the immune system of jawed vertebrates.
        Lymphocytes include \gls{bu:tcell}, and \gls{bu:bcell}.
        These cells work together in the adaptive immune response to generate antibodies against influenza
    }
}
\newglossaryentry{bu:cd8pos}
{
    type=bus,
    name=CD8+ T-cell,
    description={
        A cytotoxic T cell (also known as CD8+ T-cell) is a \gls{bu:tcell} that kills cancer cells, cells that are infected (particularly with viruses), or cells that are damaged in other ways.
        It does so by recognizing specific part of \gls{bu:antigen} and then starting a process that kills the targetted cell
    }
}
\newglossaryentry{bu:cd4pos}
{
    type=bus,
    name=CD4+ T-cell,
    description={
        The T helper cells, also known as CD4+ cells, "help" the activity of other immune cells by releasing \gls{bu:cytokine}s.
        These cells help to polarize the immune response into the appropriate kind depending on the nature of the immunological insult (e.g. virus vs. bacterium)
    }
}
\newglossaryentry{bu:cytokine}
{
    type=bus,
    name=cytokine,
    description={
        Cytokines are a broad and loose category of small proteins important in cell signaling that bind to receptor protein on the outside of (immune) cells to fulfill their signal function
    }
}
\newglossaryentry{bu:pbmc}
{
    type=bus,
    name=PBMC,
    description={
        A peripheral blood mononuclear cell is any peripheral blood cell having a round nucleus.
        These cells consist of \gls{bu:lymphocyte} and \gls{bu:monocyte}s
    },
    first={peripheral blood mononuclear cell (PBMC)}
}
\newglossaryentry{bu:bcell}
{
    type=bus,
    name=B-cell,
    description={
        B-cells produce antibody molecules; however, these antibodies are not secreted.
        Rather, they are presented on the outside of the cell where they serve as a part of B-cell receptors.
        When a B-cell is activated by an antigen, it proliferates and differentiates into an antibody-secreting effector cell, known as a plasmablast or plasma cell
    }
}
\newglossaryentry{bu:monocyte}
{
    type=bus,
    name=monocyte,
    description={
        Monocytes are a type of white blood cell.
        Monocytes and their macrophage and dendritic cell progeny serve three main functions in the immune system.
        These are phagocytosis, antigen presentation, and cytokine production.
        Phagocytosis is the process of uptake of microbes and particles followed by digestion and destruction of this material
    }
}
\newglossaryentry{bu:hai}
{
    type=bus,
    name=HAI,
    description={
        The \acrlong{ha} inhibition assay is used to measure the \gls{bu:titer} of \gls{bu:antibody} against a strain of influenza virus present in the serum.
        Antibody levels are measured before vaccination and 28 days after.
        The antibody levels are used to compute the seroprotection and seroconversion criteria
    },
    first={\acrlong{ha} inhibition assay (HAI)}
}
\newglossaryentry{bu:cmv}
{
    type=bus,
    name=CMV,
    description={
        Cytomegalovirus (CMV) is a common herpesvirus found in humans.
        Like other herpesviruses, it is a life-long infection that remains in a latent state inside the human body, until it is 'reactivated' by appropriate conditions.
        Thought to accelerate aging of the immune system and thereby impairing influenza vaccine response  \citep{van_den_Berg_2019}
    },
    first={cytomegalovirus (CMV)}
}
\newglossaryentry{bu:ebv}
{
    type=bus,
    name=EBV,
    description={
        The Epstein–Barr virus (EBV), is one of the nine known human herpesvirus types in the herpes family, and is one of the most common viruses in humans
    },
    first={Epstein-Barr virus (EBV)}
}
\newglossaryentry{bu:seropc}
{
    type=bus,
    name=seroconversion and seroprotection,
    description={
        A vaccine is considered succesful if the recipient seroconverted (4-fold or greater rise in antibody against virus after vaccination) and were seroprotected (\acrshort{gmt} \(\ge\) 40) after vaccination
    }
}
\newglossaryentry{bu:stat}
{
    type=bus,
    name=STAT,
    description={
        The signal transducer and activator of transcription (STAT) are transcription factors that work via JAK/STAT pathway regulating the expression of genes involved in cell survival, proliferation, differentiation, development, immune response, and, among other essential biological functions, hematopoiesis
    },
    first={signal transducers and activators of transcription (STAT)}
}



\newglossaryentry{d:model}
{
    type=dm,
    name=model,
    description={model is a model}
}
\newglossaryentry{d:flup}
{
    type=dm,
    name=FluPrint,
    description={Data used in this work}
}
\newglossaryentry{d:simon}
{
    type=dm,
    name=SIMON,
    description={Follow up study used in this work}
}


\begin{document}
\MyTitle{Bussiness Understanding Report}
\tableofcontents
\printglossary[type=bus]
\printglossary[type=dm]

\section{Main papers that will be used in this work}

\Gls{latex} is not cool. It never works.

A \gls{model} is a model.

\begin{itemize}
    \item the fluprint database paper \cite{tomicFluPRINTDatasetMultidimensional2019}
    \item other papers
\end{itemize}

\section{Influenza mortality papers}

\subsection{US mortality associated influenza abstract}
\cite{thompsonMortalityAssociatedInfluenza2003}
Context  Influenza and respiratory syncytial virus (RSV) cause substantial
morbidity and mortality. Statistical methods used to estimate deaths in the
United States attributable to influenza have not accounted for RSV circulation.

Objective  To develop a statistical model using national mortality and viral
surveillance data to estimate annual influenza- and RSV-associated deaths in
the United States, by age group, virus, and influenza type and subtype.

Design, Setting, and Population  Age-specific Poisson regression models using
national viral surveillance data for the 1976-1977 through 1998-1999 seasons
were used to estimate influenza-associated deaths. Influenza- and
RSV-associated deaths were simultaneously estimated for the 1990-1991 through
1998-1999 seasons.

Main Outcome Measures  Attributable deaths for 3 categories: underlying
pneumonia and influenza, underlying respiratory and circulatory, and all
causes.

Results  Annual estimates of influenza-associated deaths increased
significantly beween the 1976-1977 and 1998-1999 seasons for all 3 death
categories (P<.001 for each category). For the 1990-1991 through 1998-1999
seasons, the greatest mean numbers of deaths were associated with influenza
A(H3N2) viruses, followed by RSV, influenza B, and influenza A(H1N1). Influenza
viruses and RSV, respectively, were associated with annual means (SD) of 8097
(3084) and 2707 (196) underlying pneumonia and influenza deaths, 36 155 (11
055) and 11 321 (668) underlying respiratory and circulatory deaths, and 51 203
(15 081) and 17 358 (1086) all-cause deaths. For underlying respiratory and
circulatory deaths, 90\% of influenza- and 78\% of RSV-associated deaths occurred
among persons aged 65 years or older. Influenza was associated with more deaths
than RSV in all age groups except for children younger than 1 year. On average,
influenza was associated with 3 times as many deaths as RSV.

Conclusions  Mortality associated with both influenza and RSV circulation
disproportionately affects elderly persons. Influenza deaths have increased
substantially in the last 2 decades, in part because of aging of the
population, underscoring the need for better prevention measures, including
more effective vaccines and vaccination programs for elderly persons.

Influenza infections result in substantial morbidity and mortality nearly every
year1,2 and estimates of this burden have played a pivotal role in formulating
influenza vaccination policy in the United States.3 However, numbers of deaths
attributable to influenza are difficult to estimate directly because influenza
infections typically are not confirmed virologically or specified on hospital
discharge forms or death certificates. In addition, many influenza-associated
deaths occur from secondary complications when influenza viruses are no longer
detectable.4,5 Nonetheless, wintertime influenza epidemics have been shown to
be associated with increased hospitalizations and mortality for many diagnoses,
including congestive heart failure, chronic obstructive pulmonary disease,
pneumonia, and bacterial superinfections.6-9

Respiratory syncytial virus (RSV) epidemics often overlap with influenza
epidemics,8,10 and RSV infections have been associated with substantial
morbidity and mortality in young children and more recently in older
adults.10-14 Like influenza, RSV infections can precipitate both cardiac and
pulmonary complications.15-17 Respiratory syncytial virus infections are rarely
diagnosed in adults, in part because available rapid antigen-detection tests
are insensitive in adults and few tests for RSV are requested for this age
group by medical practitioners.16,18 It is likely that some deaths previously
attributed to influenza are actually associated with RSV infection.13,14,19

In this study, we provide age-specific estimates of deaths attributable to
influenza, by virus type and subtype, and to RSV using Poisson regression
models that incorporates national respiratory viral surveillance data. Recent
deliberations of the Advisory Committee on Immunization Practices (ACIP)
regarding influenza vaccination recommendations3 guided our choice of age
groups for these analyses.

\subsection{England influenza mortality}
\cite{greenMortalityAttributableInfluenza2013}
Very different influenza seasons have been observed from 2008/09-2011/12 in
England and Wales, with the reported burden varying overall and by age group.
The objective of this study was to estimate the impact of influenza on
all-cause and cause-specific mortality during this period. Age-specific
generalised linear regression models fitted with an identity link were
developed, modelling weekly influenza activity through multiplying clinical
influenza-like illness consultation rates with proportion of samples positive
for influenza A or B. To adjust for confounding factors, a similar activity
indicator was calculated for Respiratory Syncytial Virus. Extreme temperature
and seasonal trend were controlled for. Following a severe influenza season in
2008/09 in 65+yr olds (estimated excess of 13,058 influenza A all-cause
deaths), attributed all-cause mortality was not significant during the 2009
pandemic in this age group and comparatively low levels of influenza A
mortality were seen in post-pandemic seasons. The age shift of the burden of
seasonal influenza from the elderly to young adults during the pandemic
continued into 2010/11; a comparatively larger impact was seen with the same
circulating A(H1N1)pdm09 strain, with the burden of influenza A all-cause
excess mortality in 15–64 yr olds the largest reported during 2008/09–2011/12
(436 deaths in 15–44 yr olds and 1,274 in 45–64 yr olds). On average, 76\% of
seasonal influenza A all-age attributable deaths had a cardiovascular or
respiratory cause recorded (average of 5,849 influenza A deaths per season),
with nearly a quarter reported for other causes (average of 1,770 influenza A
deaths per season), highlighting the importance of all-cause as well as
cause-specific estimates. No significant influenza B attributable mortality was
detected by season, cause or age group. This analysis forms part of the
preparatory work to establish a routine mortality monitoring system ahead of
introduction of the UK universal childhood seasonal influenza vaccination
programme in 2013/14.

\subsection{Global mortality paper}
\cite{iulianoEstimatesGlobalSeasonal2018}
Background
Estimates of influenza-associated mortality are important for national and
international decision making on public health priorities. Previous estimates
of 250.000 500.000 annual influenza deaths are outdated. We updated the
estimated number of global annual influenza-associated respiratory deaths using
country-specific influenza-associated excess respiratory mortality estimates
from 1999–2015.
Methods
We estimated country-specific influenza-associated respiratory excess mortality
rates (EMR) for 33 countries using time series log-linear regression models
with vital death records and influenza surveillance data. To extrapolate
estimates to countries without data, we divided countries into three analytic
divisions for three age groups (<65 years, 65-74 years, and >=75 years) using
WHO Global Health Estimate (GHE) respiratory infection mortality rates. We
calculated mortality rate ratios (MRR) to account for differences in risk of
influenza death across countries by comparing GHE respiratory infection
mortality rates from countries without EMR estimates with those with estimates.
To calculate death estimates for individual countries within each age-specific
analytic division, we multiplied randomly selected mean annual EMRs by the
country's MRR and population. Global 95\% credible interval (CrI) estimates were
obtained from the posterior distribution of the sum of country-specific
estimates to represent the range of possible influenza-associated deaths in a
season or year. We calculated influenza-associated deaths for children younger
than 5 years for 92 countries with high rates of mortality due to respiratory
infection using the same methods.
Findings
EMR-contributing countries represented 57\% of the global population. The
estimated mean annual influenza-associated respiratory EMR ranged from 0.1 to
6.4 per 100.000 individuals for people younger than 65 years, 2.9 to 44.0 per
100.000 individuals for people aged between 65 and 74 years, and 17.9 to 223.5
per 100.000 for people older than 75 years. We estimated that 291 243–645 832
seasonal influenza-associated respiratory deaths (4.0–8.8 per 100.000
individuals) occur annually. The highest mortality rates were estimated in
sub-Saharan Africa (2.8–16.5 per 100 000 individuals), southeast Asia (3.5-9.2
per 100.000 individuals), and among people aged 75 years or older (51.3-99.4
per 100.000 individuals). For 92 countries, we estimated that among children
younger than 5 years, 9243-105 690 influenza-associated respiratory deaths
occur annually.
Interpretation
These global influenza-associated respiratory mortality estimates are higher
than previously reported, suggesting that previous estimates might have
underestimated disease burden. The contribution of non-respiratory causes of
death to global influenza-associated mortality should be investigated.

\section{Vaccine success criteria}
\cite{zhouHospitalizationsAssociatedInfluenza2012}

Background. Age-specific comparisons of influenza and respiratory syncytial
virus (RSV) hospitalization rates can inform prevention efforts, including
vaccine development plans. Previous US studies have not estimated jointly the
burden of these viruses using similar data sources and over many seasons.

Methods. We estimated influenza and RSV hospitalizations in 5 age categories
(<1, 1–4, 5–49, 50–64, and >=65 years) with data for 13 states from 1993–1994
through 2007–2008. For each state and age group, we estimated the contribution
of influenza and RSV to hospitalizations for respiratory and circulatory
disease by using negative binomial regression models that incorporated weekly
influenza and RSV surveillance data as covariates.

Results. Mean rates of influenza and RSV hospitalizations were 63.5 (95\%
confidence interval [CI], 37.5–237) and 55.3 (95\% CI, 44.4–107) per 100000
person-years, respectively. The highest hospitalization rates for influenza
were among persons aged >=65 years (309/100000; 95\% CI, 186–1100) and those aged
<1 year (151/100000; 95\% CI, 151–660). For RSV, children aged <1 year had the
highest hospitalization rate (2350/100000; 95\% CI, 2220–2520) followed by those
aged 1–4 years (178/100000; 95\% CI, 155–230). Age-standardized annual rates per
100000 person-years varied substantially for influenza (33–100) but less for
RSV (42–77).

Conclusions. Overall US hospitalization rates for influenza and RSV are
similar; however, their age-specific burdens differ dramatically. Our estimates
are consistent with those from previous studies focusing either on influenza or
RSV. Our approach provides robust national comparisons of hospitalizations
associated with these 2 viral respiratory pathogens by age group and over time.


\bibliographystyle{unsrt}
\bibliography{../references.bib}
\end{document}
